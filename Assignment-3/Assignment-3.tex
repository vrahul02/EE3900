\documentclass[journal,12pt,twocolumn]{IEEEtran}
\usepackage{setspace}
\usepackage{gensymb}
\singlespacing
\usepackage[cmex10]{amsmath}
\usepackage{amsthm}
\usepackage{mathrsfs}
\usepackage{txfonts}
\usepackage{stfloats}
\usepackage{bm}
\usepackage{cite}
\usepackage{cases}
\usepackage{subfig}
\usepackage{longtable}
\usepackage{multirow}
\usepackage{enumitem}
\usepackage{mathtools}
\usepackage{steinmetz}
\usepackage{tikz}
\usepackage{circuitikz}
\usepackage{verbatim}
\usepackage{tfrupee}
\usepackage[breaklinks=true]{hyperref}
\usepackage{graphicx}
\usepackage{tkz-euclide}
\usetikzlibrary{calc,math}
\usepackage{listings}
\usepackage{color}                                            %%
\usepackage{array}                                            %%
\usepackage{longtable}                                        %%
\usepackage{calc}                                             %%
\usepackage{multirow}                                         %%
\usepackage{hhline}                                           %%
\usepackage{ifthen}                                           %%
\usepackage{lscape}     
\usepackage{multicol}
\usepackage{chngcntr}
\usepackage{mathtools}

\DeclarePairedDelimiter\abs{\lvert}{\rvert}%
\DeclarePairedDelimiter\norm{\lVert}{\rVert}%

% Swap the definition of \abs* and \norm*, so that \abs
% and \norm resizes the size of the brackets, and the 
% starred version does not.
\makeatletter
\let\oldabs\abs
\def\abs{\@ifstar{\oldabs}{\oldabs*}}
%
\let\oldnorm\norm
\def\norm{\@ifstar{\oldnorm}{\oldnorm*}}
\makeatother
\DeclareMathOperator*{\Res}{Res}
\renewcommand\thesection{\arabic{section}}
\renewcommand\thesubsection{\thesection.\arabic{subsection}}
\renewcommand\thesubsubsection{\thesubsection.\arabic{subsubsection}}
\renewcommand\thesectiondis{\arabic{section}}
\renewcommand\thesubsectiondis{\thesectiondis.\arabic{subsection}}
\renewcommand\thesubsubsectiondis{\thesubsectiondis.\arabic{subsubsection}}
\hyphenation{op-tical net-works semi-conduc-tor}
\def\inputGnumericTable{}                                 %%
\lstset
{
%language=C,
frame=single, 
breaklines=true,
columns=fullflexible
}
\begin{document}
\newcommand{\comb}[2]{{}^{#1}\mathrm{C}_{#2}}
\newcommand{\BEQA}{\begin{eqnarray}}
\newcommand{\EEQA}{\end{eqnarray}}
\newcommand{\define}{\stackrel{\triangle}{=}}
\bibliographystyle{IEEEtran}
\raggedbottom
\setlength{\parindent}{0pt}
\providecommand{\mbf}{\mathbf}
\providecommand{\pr}[1]{\ensuremath{\Pr\left(#1\right)}}
\providecommand{\qfunc}[1]{\ensuremath{Q\left(#1\right)}}
\providecommand{\sbrak}[1]{\ensuremath{{}\left[#1\right]}}
\providecommand{\lsbrak}[1]{\ensuremath{{}\left[#1\right.}}
\providecommand{\rsbrak}[1]{\ensuremath{{}\left.#1\right]}}
\providecommand{\brak}[1]{\ensuremath{\left(#1\right)}}
\providecommand{\lbrak}[1]{\ensuremath{\left(#1\right.}}
\providecommand{\rbrak}[1]{\ensuremath{\left.#1\right)}}
\providecommand{\cbrak}[1]{\ensuremath{\left\{#1\right\}}}
\providecommand{\lcbrak}[1]{\ensuremath{\left\{#1\right.}}
\providecommand{\rcbrak}[1]{\ensuremath{\left.#1\right\}}}
\theoremstyle{remark}
\newtheorem{rem}{Remark}
\newcommand{\sgn}{\mathop{\mathrm{sgn}}}
\providecommand{\abs}[1]{\vert#1\vert}
\providecommand{\res}[1]{\Res\displaylimits_{#1}} 
\providecommand{\norm}[1]{\lVert#1\rVert}
%\providecommand{\norm}[1]{\lVert#1\rVert}
\providecommand{\mtx}[1]{\mathbf{#1}}
\providecommand{\mean}[1]{E[#1]}
\providecommand{\fourier}{\overset{\mathcal{F}}{ \rightleftharpoons}}
%\providecommand{\hilbert}{\overset{\mathcal{H}}{ \rightleftharpoons}}
\providecommand{\system}{\overset{\mathcal{H}}{ \longleftrightarrow}}
%\newcommand{\solution}[2]{\textbf{Solution:}{#1}}
\newcommand{\solution}{\noindent \textbf{Solution: }}
\newcommand{\cosec}{\,\text{cosec}\,}
\providecommand{\dec}[2]{\ensuremath{\overset{#1}{\underset{#2}{\gtrless}}}}
\newcommand{\myvec}[1]{\ensuremath{\begin{pmatrix}#1\end{pmatrix}}}
\newcommand{\mydet}[1]{\ensuremath{\begin{vmatrix}#1\end{vmatrix}}}
\numberwithin{equation}{subsection}
\makeatletter
\@addtoreset{figure}{problem}
\makeatother
\let\StandardTheFigure\thefigure
\let\vec\mathbf
\renewcommand{\thefigure}{\theproblem}
\def\putbox#1#2#3{\makebox[0in][l]{\makebox[#1][l]{}\raisebox{\baselineskip}[0in][0in]{\raisebox{#2}[0in][0in]{#3}}}}
\def\rightbox#1{\makebox[0in][r]{#1}}
\def\centbox#1{\makebox[0in]{#1}}
\def\topbox#1{\raisebox{-\baselineskip}[0in][0in]{#1}}
\def\midbox#1{\raisebox{-0.5\baselineskip}[0in][0in]{#1}}
\vspace{3cm}


\title{EE3900 Assignment-3}
\author{V Rahul - AI20BTECH11030}
\maketitle
\newpage
\bigskip
\renewcommand{\thefigure}{\theenumi}
\renewcommand{\thetable}{\theenumi}
Download all python codes from 
\begin{lstlisting}
    https://github.com/vrahul02/EE3900/tree/main/Assignment-3/Codes
\end{lstlisting}
%
and latex-tikz codes from 
%
\begin{lstlisting}
    https://github.com/vrahul02/EE3900/tree/main/Assignment-3/Assignment-3.tex
\end{lstlisting}
\section*{Problem Ramsey Tangent and Normal Q.20}
Find the condition that the line\\
$\myvec{l & m}\vec{x}+n=0$
should touch the circle $\quad \norm{x-\myvec{a\\b}}=r$.
\section*{Solution}
The vector equation of a line can be expressed as 
\begin{align}
    \vec{x} = \vec{q} +\mu\vec{m}
\end{align}
If $\vec{n}$ is the normal vector of a line, equation of that line can be written as 
\begin{align}
    \vec{n}^T\vec{x} = c 
\end{align}
Here
\begin{align}
    \vec{n} &= \myvec{l \\ m}\\
    c&=-n
\end{align}
The general equation of a second degree can be expressed as :
\begin{align}
    \vec{x}^T\vec{V}\vec{x}+2\vec{u}^T\vec{x}+f=0\label{1}
\end{align}
Here
\begin{align}
    \vec{u}&=\myvec{-a \\ -b}\\
    f&=a^2+b^2-r^2
\end{align}
The point of contact $\vec{q}$ of a line with a normal vector $\vec{n}$ to the conic in \eqref{1} is given by:
\begin{align}
    \vec{q} &= \vec{V}^{-1}\brak{\kappa \vec{n}-\vec{u}}\\
    \kappa &= \pm \sqrt{\frac{\vec{u}^T\vec{V}^{-1}\vec{u}-f}{\vec{n}^T\vec{V}^{-1}\vec{n}}} \label{eq:solutions/4/2/8/eq4}
\end{align}
We know that, for a circle, 
\begin{align}
    \vec{V} = \vec{I}
\end{align}
and from the properties of an Identity matrix, 
\begin{align}
    \vec{I}^{-1} &= \vec{I} \\
    \vec{I}\vec{X} &= \vec{X}   
\end{align}
Solving for the point of contact using the above equations we get,
\begin{align}
    \kappa &= \pm \sqrt{\frac{\myvec{ -a & -b}\myvec{-a \\ -b} - a^2-b^2+r^2}{\myvec{l & m }\myvec{l \\ m }}} \\
    &= \pm \frac{r}{\sqrt{l^2+m^2}} \\
    \vec{q}&=\pm\frac{r}{\sqrt{l^2+m^2}}\myvec{l\\m}+\myvec{a\\b}\\
\end{align}
Since point of contact $\vec{q}$ lies on tangent it satisfies the line equation of tangent
\begin{align}
    \vec{n}^T\vec{q} &= c\\
    \myvec{l&m}\times \left(
    \pm\frac{r}{\sqrt{l^2+m^2}}\myvec{l\\m}+\myvec{a\\b}\right)&=-n\\
    \pm r\sqrt{l^2+m^2}+al+bm &= -n\\
    \pm r\sqrt{l^2+m^2} = -n-al-bm
\end{align}
On squaring both sides
\begin{align}
    r^2(l^2+m^2)=(n+al+bm)^2
\end{align} 
This is the condition of tangency
\end{document}